\section{Assignment description}
The second assignment of Modelling and Control of Manipulators focuses on manipulators' geometry and direct kinematics. 

\begin{itemize}
    \item Download the .zip file from the Aulaweb page of this course.
    \item Implement the code to solve the exercises on MATLAB by filling the template classes called \textit{geometricModel} and \textit{kinematicModel}
    \item Write a report motivating your answers, following the predefined format on this document.
\end{itemize}

\subsection{Exercise 1}
 Given the following CAD model of an industrial 7 DoF manipulator:
 
\textbf{Q1.1} Define all the model transformation matrices, by filling the structures in the \textit{BuildTree()} function. Be careful to define the z-axis coinciding with the joint rotation axis, such that the positive rotation is the same as showed in the CAD model you received; and the x-axis, when possible, pointing to the next link. Draw on the CAD model the reference frames for each link and insert it into the report.

\textbf{Q1.2} Implement the method of \textit{geometricModel} called \textit{updateDirectGeometry()} which computes the model transformation matrices as a function of the joint position $q$. Explain the method used and comment on the results obtained.

\textbf{Q1.3} Implement the method of \textit{geometricModel} called \textit{getTransformWrtBase()} which computes the transformation matrix from the base to a given frame. Using this method, compute the following transformation matrices: $^b_e T$, $^6_2 T$. Explain the method used and comment on the results obtained.

\textbf{Q1.4} Run the MATLAB section Q1.4 in the given script \textit{main.m}, which computes the configurations of the manipulators moving from an initial joint position $q_i = [\frac{\pi}{4}, -\frac{\pi}{4}, 0, -\frac{\pi}{4}, 0, 0.15, \frac{\pi}{4}]^{\top}$ to $q_f=[\frac{5\pi}{12}, -\frac{\pi}{4}, 0, -\frac{\pi}{4}, 0, 0.18, \frac{\pi}{5}]^{\top}$. The script generates plots using methods from the given class \textit{plotManipulators}, which must not be modified. Check that it behaves reasonably and show the obtained plots in the report.

\textbf{Q1.5} Implement the method of \textit{kinematicModel} called \textit{getJacobianOfLinkWrtBase()} which takes as argument the link number, and returns the Jacobian of the corresponding link with respect to the base. Compute the Jacobian of \textbf{link 6} with respect to base for the final configuration $q_f$ given in Q1.4. Explain the method used and comment on the result obtained.

\textbf{Q1.6} Implement the method of \textit{kinematicModel} called \textit{updateJacobian()} which updates the Jacobian matrix of the manipulator for the end-effector. Compute the Jacobian for the final configuration $q_f$ given in Q1.4. Explain the method used and comment on the results obtained.

\textbf{Q1.7} Given the following joint position, $q = [0.7,-0.1,1,-1,0,0.03,1.3]^{\top}$ expressed in radians and meters, and the following joint velocities $\dot{q}=[0.9, 0.1, -0.2, 0.3, -0.8, 0.5, 0]^{\top}$ expressed in rad/s and m/s, compute the velocities of the end effector with respect to the base projected on the end effector frame, ${}^ev_{e/b}$ and ${}^e\omega_{e/b}$, using \textbf{forward kinematics}.


\textit{Remark:} All the methods must be implemented for a generic serial manipulator. For instance, joint types, and the number of joints should be parameters.

\newpage

\begin{figure} [ht]
\centering
\includegraphics[width=\textwidth*8/10]{Resources/cad.pdf}
\caption{CAD model of the robot. The directions of motion of each joint are indicated by the blue arrows.}
\label{fig:ex2}
\end{figure}

\newpage